\section{Conclusiones}

# Propiedades del algoritmo

* Se puede usar cualquier métrica $d(,)$, por ejemplo distancia euclidiana o la distancia sobre el camino en algún mapa.
* El algoritmo puede entregar varias mejores soluciones si no sólo se busca las mejor de entre las **pools** sino una cantidad específica de mejores, estas soluciones, dependiendo `POOL_SIZE` serán significativamente diferentes entre sí, lo que permite, en un caso de la vida real, descartar soluciones mejores si tienen problemas no considerados en el modelo. Para no empeorar la calidad de la solución por un POOL_SIZE bajo, se puede aplicar un proceso de reducción a las soluciones de cada tamaño antes de presentar los resultados.
* También se puede utilizar con una función objetivo $\Phi$ diferente pero parecida, por ejemplo una concostos fijos diferentes para cada ubicación.
% TODO: ¿No afectaría esto el esquema de reducción?

% TODO: Discusión sobre métodos que reciben initial guess y después iteran.

% TODO: Nombrar esta metodología de extensión-reducción, explicar utilidades en otros campos, ¿Nombrar semejanza con método de implicancias en búsqueda de patrones?

% TODO: Considerar expansión más "elástica", al agregar una nueva instalación a un conjunto se pueden empujar las otras, (osea, aplicar un optimizador local aquí, (¿Comportamiento de burbujas?).
