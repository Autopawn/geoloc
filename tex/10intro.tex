\section{Introducción}

El campo del \emph{location analysis} trata sobre la decisión de donde ubicar instalaciones en un espacio determinado, en dicho espacio generalmente se encuentran \emph{clientes} relacionados con los objetivos que buscados, determinados en la función objetivo. Algunos criterios de clasificación de estos problemas son los siguientes\cite{revelle2005location}:

\begin{itemize}
\item \textbf{Location problems} y \textbf{layout problems}, en los primeros el alcance de las instalaciones es pequeño, en los segundos es grande y por lo tanto las interacciones entre estas se convierten en norma.
\item En \textbf{espacio $d$-dimensional} o sobre una \textbf{red o grafo}.
\item \textbf{Continuos} o \textbf{discretos}, dependiendo de si las instalaciones se pueden poner en un espacio continuo o en posiciones fijas, la dificultad de los primeros suele estar en la no linealidead, mientras que la de los segundos suele estar en la explosión combinatoria.
\item Si se tienen \textbf{pull objectives}, \textbf{push objectives} o \textbf{balancing objectives}, dependiendo de si se busca que las instalaciones queden cerca (minimizar el costo de transporte) o lejos de los clientes, o en el caso de los \emph{balancing objectives} intentar que se logre equidad respecto a las distancias de los clientes a cada instalación.
\item Problemas de \textbf{$1$ instalación}, \textbf{$p$ instalaciones} o de una \textbf{cantidad no predeterminada de instalaciones}, principalmente modelos \emph{free-entry} que corresponden a problemas donde se debe decidir el número más conveniente de instalaciones.
\end{itemize}

% TODO: Asegurarse de que free-entry no se refiere a los problemas greedy.

En el presente se muestra un problema general que cae en la categoría de los \emph{layout problems}, \emph{discretos}, con \emph{pull objectives}, y de \emph{free-entry}, cabe señalar que no es necesario alcanzar todos los clientes si estos no contribuyen a aumentar la función objetivo. El modelo utilizado puede ajustarse tanto a \emph{espacios $d$-dimensionales} como \emph{redes}, mientras que se trate de un \emph{espacio métrico}.

% El modelo presentado es suficientemente general para aplicar a varios problemas de la vida real.

Se presenta una forma de obtener una aproximación a la solución óptima de elegir un subconjunto de ubicaciones de instalaciones que maximiza la ganancia obtenida por conectar unos clientes fijos y considerando un posible costo de transporte para alcanzarlos, puesto que obtenerla es un problema \emph{NP-completo}.
% TODO: Encontrar referencia sobre NP-completitud (el location analysis synthesis survey lo señala sobre un modelo parecido, ¿Seguir referencia?)

% TODO: Señalar que el location analysis es un campo mucho más grande que las clasificaciones aquí presentadas.

% TODO: Indicar cómo se aborda el trabajo.

% TODO: En general, ¿Análisis de sensisiblidad de las soluciones?
