\section{Introducción}

El campo del \emph{location analysis} trata sobre la decisión de donde ubicar instalaciones en un espacio determinado, algunos criterios de clasificación de estos problemas son los siguientes\cite{revelle2005location}:

\begin{itemize}
\item \textbf{Location problems} y \textbf{layout problems}:
\item En \textbf{espacio $d$-dimensional} o sobre una \textbf{red o grafo}.
\item \textbf{Continuos} o \textbf{discretos}, dependiendo de si las instalaciones se pueden poner en un espacio continuo o en posiciones fijas (problemas \emph{uno-cero}).
\end{itemize}

El algoritmo aquí presentado trata los \emph{layout problems} y problemas \emph{discretos}, y puede usarse tanto ante \emph{espacios $d$-dimensionales} o \emph{redes}, mientras que se trate de un \emph{espacio métrico}. Se presenta una forma de obtener una aproximación a la solución óptima de elegir un subconjunto de ubicaciones de instalaciones que maximiza la ganancia obtenida por conectar unos clientes fijos y considerando un posible costo de transporte para alcanzarlos, puesto que obtenerla es un problema \emph{NP-completo}.
% TODO: Encontrar referencia sobre NP-completitud.

% TODO: Indicar cómo se aborda el trabajo.
