\section{Métricas de disimilitud}
\label{metricas_disimilitud}

Como se señaló, previamente para comparar soluciones se requiere una medida de disimilitud, siendo lógico que esté asociada a la métrica de distancia $d(a,b)$ entre dos ubicaciones $a,b$. Para que una medida $D(A,B)$ entre dos soluciones sea una métrica debe cumplir con:

\begin{enumerate}
\item $D(A,B) = 0 \Leftrightarrow A=B$
\item $D(A,B) = D(B,A)$
\item $D(A,C) \leq D(A,B) + D(B,C)$
\end{enumerate}

Se proponen dos métricas de disimilitud, la primera corresponde al \emph{minimum weight matching}, una variante de la \emph{bottleneck distance}, la otra corresponde al \emph{mean geometric error} de Lindstrom y Turk. Ambas fueron definidas para comparar imágenes binarias.
% ^ Todo en página 360 de la enciclopedia de distancias.
% TODO: ¿Considerar la distancia de Hausdorff modificada?

La razón de la elección de estas métricas es que son afectadas por todos los elementos de ambos conjuntos, el desplazamiento de un elemento siempre afectará la disimilitud calculada (a menos que dicho elemento se desplace en un isocontorno de dicha función), lo que hace que cualquier diferencia entre combinaciones de posibles ubicaciones sea considerada en el algoritmo. Si se utilizada, por ejemplo, \emph{simple linkage} o \emph{complete linkage} que sólo dependen del par de ubicaciones más cercano o más alejado respectivamente, sería común que dos soluciones, pese a ser diferentes, tengan la misma disimilitud a una tercera, produciendo empates que se rompen arbitrariamente en el proceso de reducción.

\subsection{Minimum weight matching}

Esta métrica de disimilitud entre dos soluciones de un mismo tamaño $n$, $A = \{a_1,a_2,...,a_n\}$ y $B = \{b_1,b_2,...,b_n\}$, se define como:

\begin{equation}
D(A,B) = \inf_{m \in M}\left( \sum_{(a,b) \in m} d(a,b)\right)
\end{equation}

Donde $M$ es el conjunto de todos los posibles matchings completos entre $A$ y $B$.

Esta métrica corresponde a la suma de las distancias entre los pares del matching completo $m^\star$ entre $A$ y $B$ que minimiza esta suma de distancias. $m^\star$ es, por lo tanto, la solución a un problema de asignación, el cual se puede resolver utilizando el \emph{algoritmo de Kuhn–Munkres} o \emph{método húngaro}.

\subsection{Mean geometric error}

Esta métrica de disimilitud entre dos conjuntos de ubicaciones, $A = \{a_1,a_2,...,a_n\}$ y $B = \{b_1,b_2,...,b_m\}$, se define como:

\begin{equation}
D(A,B) = \frac{1}{|A|+|B|}\left(\sum_{a \in A}\inf_{b \in B}d(a,b) +
\sum_{b \in B}\inf_{a \in A}d(b,a)\right)
\end{equation}

Esta métrica corresponde a la distancia promedio que tienen las ubicaciones de ambos conjuntos a la ubicación más cercana en el conjunto opuesto. Respecto a la métrica anterior, tiene la ventaja de que puede calcularse en $O(n^2)$ en vez de $O(n^3)$.

\subsection{Justificación del método de reducción con rango de visión}
\label{justificacion_reduccion}


%
% ## Demostración de que la *minimal match distance* es una métrica
%
% Para que *minimal match distance* sea una métrica debe cumplir con:
%
% 1. $D(A,B) = 0 \Leftrightarrow A=B$
% 2. $D(A,B) = D(B,A)$
% 3. $D(A,C) \leq D(A,B) + D(B,C)$
%
% Estas propiedades se pueden demostrar si $d(,)$ es una métrica.
%
% ### Primera propiedad
%
% Si $D(A,B) = 0$, para el matching completo óptimo $m^\star$ todos los pares $(a,b)$ deben cumplir que $d(a,b)=0$ y por lo tanto $a=b$, como cada elemento de $A$ debe estar en $B$ (sino los matchings serían inválidos) y $|A|=|B|=n$, entonces $A=B$.
%
% Por otro lado, si se parte de $A=B$, se puede contruír un matching completo $m$ con pares $(a,a)$, con este matching la suma de distancias será $0$ y esta será la suma de distancias entre pares óptima, pues no puede haber una suma de distancias menor.
%
% ### Segunda propiedad
%
% Considerando que $d(a,b)=d(b,a)$, sea $m^\star$ un matching completo óptimo entre $A$ y $B$ cuya suma de distancias entre pares resulta, por lo tanto, en $D(A,B)$. Sea $m^{\star\star}$ el mismo matching pero invirtiendo los pares, este matching debe ser óptimo entre $B$ y $A$, de existir un matching mejor entre $B$ y $A$, podría ser utilizado (invirtiendo los pares) entre $A$ y $B$ y $m^\star$ ya no sería óptimo. Como la suma de las distancias entre los pares de $m^\star$ es la misma que de $m^\star\star$, y ambos son matching óptimos de $A,B$ y $B,A$, $D(A,B)=D(B,A)$.
%
% ### Tercera propiedad
%
% Sea $m^\star_1$ un matching completo óptimo entre $A$ y $B$ cuya suma de distancias entre pares corresponde a $D(A,B)$ y $m^\star_2$ un matching completo óptimo entre $B$ y $C$ cuya suma de distancias entre pares corresponde a $D(B,C)$.
%
% Dándole una numeración arbitraria a los elementos de $A$:
% $$
% A = \{a_1,a_2,...,a_n\}
% $$
%
% Se numera cada elemento de $B$ con el mismo índice que tenga el elemento de $A$ con que forme un par en el matching $m^\star_1$, de tal manera que:
% $$
% m^\star_1 = \{(a_1,b_1),(a_2,b_2),...,(a_n,b_n)\}
% $$
%
% Posteriormente, de la misma manera, se numera cada elemento de $C$ con el mismo índice que tenga el elemento de $B$ con que forme un par en el matching $m^\star_2$, de tal manera que:
% $$
% m^\star_2 = \{(b_1,c_1),(b_2,c_2),...,(b_n,c_n)\}
% $$
%
% Puesto que $d$ es una métrica cumple con:
% $$
% d(a_i,c_i) \leq d(a_i,b_i) + d(b_i,c_i)
% $$
%
% Sumando sobre $i$:
% $$
% \sum_{i=1}^n d(a_i,c_i) \leq \sum_{i=1}^n d(a_i,b_i) + \sum_{i=1}^n d(b_i,c_i)
% $$
% $$
% \sum_{i=1}^n d(a_i,c_i) \leq D(A,B) + D(B,C)
% $$
%
% Es posible construir un matching completo $m$ entre $A$ y $C$ tal que $m= \{(a_1,c_1),(a_2,c_2),...,(a_n,c_n)\}$, cuya suma de distancias entre pares será $\sum_{i=1}^n d(a_i,c_i)$. Como la suma de distancias entre pares de un matching óptimo $m^\star_3$ entre $A$ y $C$ debe ser menor o igual que la suma de distancias entre pares de $m$, se tiene:
% $$
% D(A,C) \leq \sum_{i=1}^n d(a_i,c_i)
% $$
% $$
% D(A,C) \leq \sum_{i=1}^n d(a_i,c_i) \leq D(A,B) + D(B,C)
% $$
% $$
% D(A,C) \leq D(A,B) + D(B,C)
% $$
%
